%% start of file `template_en.tex'.
%% Copyright 2007 Xavier Danaux (xdanaux@gmail.com).
%
% This work may be distributed and/or modified under the
% conditions of the LaTeX Project Public License version 1.3c,
% available at http://www.latex-project.org/lppl/.


\documentclass[11pt,a4paper]{moderncv}

% moderncv themes
%\moderncvtheme[blue]{casual}                 % optional argument are 'blue' (default), 'orange', 'red', 'green', 'grey' and 'roman' (for roman fonts, instead of sans serif fonts)
\moderncvtheme[green]{classic}                % idem

% character encoding
\usepackage[utf8]{inputenc}                   % replace by the encoding you are using

% adjust the page margins
\usepackage[scale=0.8]{geometry}
\recomputelengths                             % required when changes are made to page layout lengths

% personal data
\firstname{Zhongpeng}
\familyname{Lin}
\title{}               % optional, remove the line if not wanted
\address{F11, Building 5, No.4 Zhongguancun South 4th. Street}{100190 Beijing, China}    % optional, remove the line if not wanted
\phone{8610 8268 3088}                      % optional, remove the line if not wanted
\email{lin.zhp@gmail.com}                      % optional, remove the line if not wanted

%\nopagenumbers{}                             % uncomment to suppress automatic page numbering for CVs longer than one page


%----------------------------------------------------------------------------------
%            content
%----------------------------------------------------------------------------------
\begin{document}
\maketitle

\section{Education}
\cventry{2007--present}{MS, Computer Software and Theory (expect 2010)}{Institute of Software, Chinese Academy of Sciences}{Beijing, China}{}{}%   arguments 3 to 6 are optional
\cventry{2003--2007}{BS, Software Engineering}{Xiamen University}{Xiamen, China}{}{}  % arguments 3 to 6 are optional

\section{Research Interests}
\cvlistitem{Empirical Software Engineering}
\cvlistitem{Software Architecture}
\cvlistitem{Data Mining, Business Intelligence}

% Publications from a BibTeX file
\bibliographystyle{plain}
\bibliography{publications}       % 'publications' is the name of a BibTeX file


\section{Experience}
\subsection{Vocational}
\cventry{2008--present}{Senior Software Engineer, DBA, Techincal Mentor}{Realike (hoolai.com, a company with more 20 employees, focusing on developing social games)}{}{Beijing,China}{I am working as a part-time engineer. With other colleagues, we used Ruby On Rails (ROR) and Adobe Flex to develop several web games on several social network websites, including Facebook. Some of our games have more than 600 thousand daily active users (DAU) and more than 40 million HTTP requests per day, which is believed to be the largest ROR application in China. I also help them to train new employees on ROR.}                % arguments 3 to 6 are optional
\cventry{2007}{Intern Software Engineer}{Practical Strategies China}{Xiamen}{China}{This is a branch of a American company, Pratical Strategies, Inc. We used ROR and adopted eXtreme Programming to develop several projects, including a Agile method management system and an on-line music composition website}                % arguments 3 to 6 are optional
\subsection{Miscellaneous}
\cventry{year--year}{Job title}{Employer}{City}{}{Description line 1\newline{}Description line 2}% arguments 3 to 6 are optional

\section{Languages}
\cvlanguage{language 1}{Skill level}{Comment}
\cvlanguage{language 2}{Skill level}{Comment}
\cvlanguage{language 3}{Skill level}{Comment}

\section{Computer skills}
\cvcomputer{category 1}{XXX, YYY, ZZZ}{category 4}{XXX, YYY, ZZZ}
\cvcomputer{category 2}{XXX, YYY, ZZZ}{category 5}{XXX, YYY, ZZZ}
\cvcomputer{category 3}{XXX, YYY, ZZZ}{category 6}{XXX, YYY, ZZZ}

\section{Interests}
\cvline{hobby 1}{\small Description}
\cvline{hobby 2}{\small Description}
\cvline{hobby 3}{\small Description}

\closesection{}                   % needed to renewcommands
\renewcommand{\listitemsymbol}{-} % change the symbol for lists

\section{Extra 1}
\cvlistitem{Item 1}
\cvlistitem{Item 2}
\cvlistitem[+]{Item 3}            % optional other symbol

\section{Extra 2}
\cvlistdoubleitem[\Neutral]{Item 1}{Item 4}
\cvlistdoubleitem[\Neutral]{Item 2}{Item 5}
\cvlistdoubleitem[\Neutral]{Item 3}{}


\end{document}


%% end of file `template_en.tex'.
